\documentclass{article}
\usepackage[utf8]{inputenc}
\usepackage[spanish]{babel}
\usepackage{listings}
\usepackage{graphicx}


\begin{document}

\begin{titlepage}
    \begin{center}
        \vspace*{1cm}
            
        \Huge
        \textbf{Informe Actividad Evaluativa}
            
        \vspace{0.5cm}
        \LARGE
        Informatica II
            
        \vspace{1.5cm}
            
        \textbf{Daniela Andrea Gallego Diaz}
            
        \vfill
            
        \vspace{0.8cm}
            
        \Large
        Despartamento de Ingeniería Electrónica y Telecomunicaciones\\
        Universidad de Antioquia\\
        Medellín\\
        Marzo de 2021
            
    \end{center}
\end{titlepage}

\tableofcontents
\newpage
\section{Introduccion}\label{intro}
En esta actividad el objetivo principal es llevar determinados objetos de una posicion A a una posición B de manera exitosa y siguiendo una serie de pasos, en este caso, el desafio será llevar dos tarjetas y una hoja de block de una posición inicial a una final. 

\section{Pasos a seguir} \label{contenido}
Deberá hacer uso de una hoja de block y dos tajetas rectangulares, ambas del mismo tamaño.\\ 
IMPORTANTE: Para esta actividad tenga en cuenta que deberá utilizar solo una mano y seguir los pasos al pie de la letra. \\
Antes de comerzar enumeraremos cada lado de las tarjetas de la siguiente manera:
\begin{itemize}
    \item A los lados de menos longitud de cada tarjeta los nombraremos lado 1 y lado 2.
    \item A los lados de mayor longitud de cada tarjeta los nombraremos lado 3 y lado 4.
\end{itemize}
    ACLARACIÓN: Tenga encuenta que el proceso anterior se hace con ambas tarjetas, es decir, las dos tendrán lado 1,2,3 y 4 respectivamente.\\
Ahora bien, para la posición inicial deberás seguir estos dos simples pasos:
\begin{itemize}
    \item Las dos tarjetas estarán sobre la mesa, una encima de la otra y ubicadas en la misma posición, es decir, de manera que si las visualizas parecerá que sólo hay una tarjeta sobre la mesa. 
    \item La hoja de block irá encima de ambas tarjetas, debemos asegurarnos de que la hoja no mueva las tarjetas y las cubra por completo, por lo que la hoja se pondrá con mucho cuidado.
\end{itemize}
Ahora todo esta listo para empezar el desafío.
\subsection{Instrucciones}
\begin{enumerate}
    \item Agarre la hoja de block por uno de sus extremos y levantela con mucho cuidado dejando al descubierto las tarjetas, ponga la hoja en otro punto de la mesa sin que toque las tarjetas y asegurandose de que la hoja quede totalmente plana.
    \item Ahora agarre las tarjetas por los lados 3 y 4 asegurandose de que permanezcan juntas, es decir, una encima de la otra. Levantelas y coloquelas en posición vertical sin soltarlas de su mano. Para que queden en forma vertical asegurese de que el lado 1 de ambas tarjetas queden mirando hacia el techo y el lado 2 de ambas tarjetas queden mirando hacia la superficie de la mesa.\\
    ACLARACIÓN: Las tarjetas no deben estar tocando la mesa y hasta el momento no las puede soltar de su mano.
    \item Ahora desplace su mano, sin soltar ambas tarjetas, hacia donde se encuentra la hoja y apoye el lado 2 de ambas tarjetas sobre la superfie de la hoja.\\
    ACLARACIÓN: Sólo el lado 2 de ambas tarjetas debe estar tocando la hoja, además el lado 1 de las tarjetas debe seguir mirando hacia el techo. 
    \item Posicione su dedo pulgar sobre el lado 3, pero sólo deberá estar apoyado sobre una sola tarjeta. Luego apoye su dedo índice sobre el lado 1 de ambas tarjetas y el resto de los dedos de su mano deberán estar en el lado 4, pero sólo apoyados en una tarjeta (La misma en donde apoya su dedo pulgar).
    \item Incline las tarjetas hacia el lado contrario de la tarjeta que está sostenida por el lado 3 y 4 de esta misma, tenga en cuenta que ambas tarjetas deben permanecer juntas por el momento y que el dedo índice debe estar apoyado en el lado 1 de ambas.\\
    ACLARACIÓN: La inclinación debe ser lo más leve posible.
    \item Ahora sin dejar que los lados 1 de ambas tarjetas se separen, intente separar los demás lados, es decir, el lado 2 de una de las tarjetas debe quedar separado del lado 2 de la otra, y así mismo con los demás lados (a excepción del lado 1), y ahora apoye el lado 2 de la tarjeta que no está apoyada a la hoja y deberá ver como se va formando con ambas tarjetas una pirámide.
    \item Por último vaya separando cuidadosamente los lados 2 de las tarjetas hasta que vea claro la forma de una pirámide (No deben quedar ni muy separadas ni muy juntas), ahora busque equilibrio entre ellas para que se sostengan y cuando crea que se ha formado equilibrio, retire cuidadosamente sus dedos de las tarjetas. Si las tarjetas caen, vuelva a ponerlas en la posición que indica el paso 3 y siga de nuevo los pasos desde el 4. Si las tarjetas se quedan en pirámide, el desafío ha sido completado exitosamente. 
\end{enumerate}

\section{Casos de Prueba} \label{Pruebas}
Este documento será entregado a 3 personas, las cuales serán los casos de prueba que demostrarán que tan efectivas son las instrucciones dadas y si llevan a que el desafío sea exitoso o no. Todo el proceso que realicen estas personas será documentado a través de un video; cabe aclarar que no se les dio ningún otro tipo de instrucción a las personas de prueba, sólo contaban con el documento para guiarse.
\end{document}
